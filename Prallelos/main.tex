\documentclass[10pt,a4paper]{article}

%% Carga del paquete de formato personalizado
\usepackage{formato}

%% =====================================================
%% CONFIGURACIÓN DE LOS DATOS DEL DOCUMENTO
%% =====================================================
\documenttype{Proyecto Final -- Fase 1}
\documenttitle{Paralelización de Algoritmos Matriciales Masivos con CUDA: Warshall Lógico (Cerradura Transitiva Booleana)}
\subject{Algoritmos Paralelos y Distribuidos}
\advisor{Mgt. Ray Dueñas Jiménez}
\semester{2025-II}

%% =====================================================
%% AUTORES (ajusta si tu grupo es distinto)
%% =====================================================
\addauthor{Castro Pari, Rayneld Fidel}
\addauthor{Mayhuire Chacon, Brenda Lucia}
\addauthor{Mendoza Quispe, Jose Daniel}
\addauthor{Perez Cahuana, Gabriel}
\addauthor{Zevallos Yanqui, Andy Jefferson}

%% =====================================================
%% MACROS ÚTILES (no requieren archivos externos)
%% =====================================================
% Inclusión segura de imágenes: si no existe el archivo, muestra un recuadro.
\newcommand{\maybeincludegraphics}[2][]{%
  \IfFileExists{#2}{\includegraphics[#1]{#2}}{%
    \fbox{\parbox[c][3.2cm][c]{0.92\linewidth}{\centering \textbf{Archivo no encontrado:}\\ \texttt{#2}}}%
  }%
}

% Notación para indexar matriz plana
\newcommand{\idx}[3]{#1[#2\cdot N + #3]}

%% =====================================================
%% INICIO DEL DOCUMENTO
%% =====================================================
\begin{document}

%% =====================================================
%% PORTADA
%% =====================================================
\thispagestyle{empty}
\makeatletter
%% =====================================================
%% Portada - Documento académico - Versión APA
%% =====================================================
\begin{titlepage}
    \thispagestyle{empty}
    
    % Cambiar a interlineado sencillo solo para la portada
    \singlespacing
    
    %% =====================================================
    %% ENCABEZADO INSTITUCIONAL
    %% =====================================================
    \begin{center}
        UNIVERSIDAD NACIONAL DE SAN ANTONIO ABAD DEL CUSCO\\[0.3cm]
        FACULTAD DE INGENIERÍA ELÉCTRICA, ELECTRÓNICA, INFORMÁTICA Y MECÁNICA\\[0.3cm]
        ESCUELA PROFESIONAL DE INGENIERÍA INFORMÁTICA Y DE SISTEMAS
    \end{center}

    
    %% =====================================================
    %% ESCUDO/LOGO INSTITUCIONAL
    %% =====================================================
    \vspace{0.25cm}
    \begin{figure}[H]
        \centering
        \includegraphics[scale=0.3]{imagenes/escudo.png}
    \end{figure}
    
    %% =====================================================
    %% TÍTULO DEL DOCUMENTO
    %% =====================================================
    \begin{center}
        \Large{\textbf{\@documenttype}} \\[0.3cm]
        \rule{160mm}{0.6mm} \\[0.3cm]
        {\Large \textbf{\@documenttitle}} \\[0.1cm]
        \rule{160mm}{0.6mm}
    \end{center}
    
    \vspace{1cm}
    
    %% =====================================================
    %% INFORMACIÓN DEL CURSO Y PARTICIPANTES
    %% =====================================================
    \noindent\hspace*{2cm}%
    \parbox{0.5\textwidth}{
        \raggedright
        \textbf{Asignatura:} \\
        \hspace*{2em} \@subject \\[0.8cm]
        \textbf{Docente:} \\
        \hspace*{2em} \@advisor \\[0.8cm]
        % Condicional para singular/plural de "Estudiante(s)"
        \textbf{\ifnum\value{authorcount}>1 Estudiantes:\else Estudiante:\fi} \\
        \hspace*{2em}
        \parbox{\dimexpr\linewidth-2em\relax}{
            \raggedright
            \vspace{0.3em}
            \authorlist
        }
    }
    

    
    \vfill
    
    %% =====================================================
    %% FECHA DE PUBLICACIÓN
    %% =====================================================
    \begin{center}
        Cusco - Perú \\ \the\year
    \end{center}
    
    % Restaurar interlineado doble para el resto del documento
    \doublespacing
\end{titlepage}
\makeatother

%% =====================================================
%% PRELIMINARES
%% =====================================================
\startpreliminary
\clearpage

\tableofcontents
\clearpage

\listoffigures
\clearpage
\listoftables
\clearpage

%% =====================================================
%% PRINCIPAL
%% =====================================================
\startmain

% =====================================================

\begin{abstractenv}
La \textbf{cerradura transitiva booleana} (también llamada \textbf{Warshall lógico}) es un algoritmo clásico de teoría de grafos
y álgebra booleana que determina la \textbf{alcanzabilidad} entre todos los pares de vértices de un grafo dirigido.
Dada una matriz de adyacencia booleana $A \in \{0,1\}^{N\times N}$, el algoritmo produce (in-place) una matriz $T$ tal que
$T_{ij}=1$ si existe algún camino desde $i$ hasta $j$. Su estructura es completamente matricial, con costo temporal
\textbf{$O(N^3)$} y costo espacial \textbf{$O(N^2)$}, lo cual lo convierte en un candidato fuerte para demostrar aceleración mediante GPU (CUDA)
cuando $N$ es masivo (requisito del proyecto: $N\ge 1024$).

En esta Fase 1 se presenta: (i) fundamento teórico y representación matricial; (ii) pseudocódigo y análisis de complejidad;
(iii) una implementación secuencial en C optimizada a nivel básico (memoria contigua y atajo por $A_{ik}$); (iv) metodología de pruebas
(reproducibilidad mediante semilla, densidad controlable $p$, y medición precisa del núcleo); y (v) el diseño conceptual para la Fase 2
(CUDA), incluyendo la paralelización 2D sobre $(i,j)$ por cada iteración $k$, y consideraciones de coalescencia, sincronización y \textit{tiling}.
\keywords{Cerradura transitiva, Warshall lógico, grafos, matrices booleanas, $O(N^3)$, CUDA, paralelización, rendimiento}
\end{abstractenv}
\clearpage

% =====================================================
\section{Introducción}
El cómputo moderno en ingeniería e informática trabaja con \textbf{matrices masivas} en múltiples contextos: conectividad de redes,
dependencias entre módulos de software, grafos de estados, análisis de rutas, relaciones jerárquicas, entre otros.
Cuando $N$ crece (por ejemplo $N\ge 1024$), algoritmos cúbicos $O(N^3)$ se vuelven costosos en CPU, motivando el uso de GPU con CUDA.

En este marco, se propone el algoritmo de \textbf{Warshall lógico} para la Fase 1 del proyecto
\textit{Paralelización de Algoritmos Matriciales Masivos con CUDA}, debido a que:
\begin{itemize}
  \item opera directamente sobre una matriz $N\times N$;
  \item tiene gran carga computacional ($O(N^3)$);
  \item su salida es verificable por comparación bit a bit;
  \item presenta paralelismo masivo en cada fase $k$ sobre las celdas $(i,j)$.
\end{itemize}

% =====================================================
\section{Objetivos}
\subsection{Objetivo general}
Diseñar, implementar y evaluar (en Fase 2) la aceleración obtenida al paralelizar en CUDA el algoritmo de
\textbf{cerradura transitiva booleana} (Warshall lógico) para matrices masivas ($N\ge 1024$), comparando contra una versión secuencial CPU.

\subsection{Objetivos específicos}
\begin{itemize}
  \item Describir formalmente el problema de alcanzabilidad all-pairs y su modelado mediante matrices booleanas.
  \item Analizar complejidad temporal $O(N^3)$ y espacial $O(N^2)$, estimando magnitudes reales para $N\ge 1024$.
  \item Implementar una versión secuencial reproducible y medible, separando correctamente \textit{núcleo} vs \textit{overhead}.
  \item Establecer una metodología de pruebas: tamaños $N$, densidad $p$, semilla, validación para casos pequeños.
  \item Definir el plan de paralelización CUDA para Fase 2: mapeo de hilos, sincronización por $k$, y optimizaciones esperadas.
\end{itemize}

% =====================================================
\section{Marco teórico}
\subsection{Grafos dirigidos, matriz de adyacencia y alcanzabilidad}
Un grafo dirigido $G=(V,E)$ con $|V|=N$ puede representarse mediante una matriz booleana de adyacencia $A$:
\[
A_{ij}=
\begin{cases}
1, & \text{si existe arco directo } i\to j\\
0, & \text{caso contrario}
\end{cases}
\]
La \textbf{alcanzabilidad} (reachability) pregunta si existe un camino (de longitud $\ge 1$) desde $i$ hasta $j$.
La \textbf{cerradura transitiva} $T$ de $A$ es una matriz tal que:
\[
T_{ij}=1 \iff \exists \text{ un camino de } i \text{ a } j.
\]
Este problema aparece en bases de datos (consultas recursivas), análisis de dependencias y redes de comunicación.

\subsection{Álgebra booleana en matrices}
Warshall lógico aplica operaciones booleanas:
\[
x \lor y \quad (\text{OR}), \qquad x \land y \quad (\text{AND}),
\]
que en implementación pueden codificarse con enteros 0/1, o con operaciones bit a bit si se usan \textit{bitsets}.
La regla central es: ``$i$ llega a $j$ si ya llegaba, o si llega a $k$ y $k$ llega a $j$''.

\subsection{Warshall lógico y relación con otros algoritmos}
Warshall (1962) formaliza propiedades sobre matrices booleanas; su algoritmo computa cerradura transitiva.
Floyd--Warshall, en cambio, computa distancias mínimas (con suma y mínimo) y comparte estructura triple-anidada.
La versión booleana conserva el patrón cúbico, lo cual es útil como base didáctica para paralelización.

% =====================================================
\section{Descripción del algoritmo (Warshall lógico)}

\subsection{Nombre del algoritmo y campo de aplicación}
El algoritmo estudiado se denomina \textbf{Warshall lógico} (o \textbf{cerradura transitiva booleana}).
Pertenece al campo de \textbf{teoría de grafos} y \textbf{álgebra booleana}, y se aplica cuando se requiere resolver
\textbf{alcanzabilidad all-pairs} (reachability) sobre un grafo dirigido representado matricialmente.

\textbf{Campos de aplicación típicos}:
\begin{itemize}
  \item \textbf{Redes y enrutamiento}: determinar si un router/nodo puede comunicarse con otro a través de rutas multi-salto.
  \item \textbf{Análisis de dependencias en software}: saber si un módulo depende directa o indirectamente de otro (grafos de llamadas).
  \item \textbf{Bases de datos y consultas recursivas}: cierre transitivo en relaciones (por ejemplo, jerarquías o grafos de referencias).
  \item \textbf{Modelado de estados y verificación}: alcanzabilidad entre estados en autómatas/grafos de transición.
  \item \textbf{Procesamiento de imágenes (modelado como grafo)}: conectividad entre regiones/píxeles al representar adyacencias como grafo.
\end{itemize}
En todos estos casos, la salida $T$ resume conectividad global y habilita consultas posteriores en tiempo $O(1)$ por par $(i,j)$.

\subsection{Recurrencia}
Sea $A$ la matriz booleana. El algoritmo actualiza:
\[
A_{ij} \leftarrow A_{ij} \lor (A_{ik}\land A_{kj}), \qquad k=0,\dots,N-1.
\]
Interpretación: al fijar un $k$, se permite usar $k$ como vértice intermedio; luego se incorpora a los intermedios permitidos.

\subsection{Descripción detallada del funcionamiento (idea e invariante)}
El algoritmo recorre un conjunto creciente de \textbf{vértices intermedios permitidos}.
En la iteración $k$, se ``habilita'' el vértice $k$ como posible intermedio en caminos de $i$ a $j$.

\textbf{Lectura operativa de la actualización}:
\[
A_{ij} \leftarrow A_{ij} \lor (A_{ik}\land A_{kj})
\]
\begin{itemize}
  \item $A_{ij}$: ya se conocía un camino (directo o descubierto previamente) de $i$ a $j$.
  \item $(A_{ik}\land A_{kj})$: existe un camino de $i$ a $k$ y de $k$ a $j$; por lo tanto existe un camino de $i$ a $j$
        usando $k$ como intermedio.
  \item El OR final acumula conocimiento: si era alcanzable antes o se vuelve alcanzable usando $k$, queda marcado como alcanzable.
\end{itemize}

\textbf{Invariante (base formal de corrección).}
Sea $A^{(k)}$ la matriz luego de procesar el valor $k$ (o equivalentemente, tras permitir intermedios en $\{0,\dots,k\}$).
El invariante es:
\[
A^{(k)}_{ij}=1 \iff \exists \text{ un camino de } i \text{ a } j \text{ cuyos vértices intermedios están en } \{0,\dots,k\}.
\]
\textbf{Justificación breve}:
\begin{itemize}
  \item \emph{Base}: para $k=-1$ (antes de iterar), $A^{(-1)}$ representa aristas directas (caminos sin intermedios).
  \item \emph{Paso inductivo}: al pasar de $k-1$ a $k$, se mantiene lo ya alcanzable y se añade lo que se vuelve alcanzable usando
        al nuevo intermedio $k$ (exactamente el término $A_{ik}\land A_{kj}$).
\end{itemize}
Al finalizar $k=N-1$, se han permitido todos los vértices como intermedios, obteniendo la cerradura transitiva completa.


\subsection{Pseudocódigo}
\begin{algorithm}
\caption{Cerradura transitiva booleana (Warshall lógico)}
\begin{algorithmic}[1]
\Require Matriz booleana $A \in \{0,1\}^{N\times N}$
\Ensure $A$ actualizada con su cerradura transitiva
\For{$k=0$ \textbf{to} $N-1$}
  \For{$i=0$ \textbf{to} $N-1$}
    \For{$j=0$ \textbf{to} $N-1$}
      \State $A[i][j] \gets A[i][j]\ \lor\ (A[i][k] \land A[k][j])$
    \EndFor
  \EndFor
\EndFor
\end{algorithmic}
\end{algorithm}

\subsection{Ejemplo controlado}
Para $N=4$ y:
\[
A=
\begin{bmatrix}
0&1&0&0\\
0&0&1&0\\
0&0&0&1\\
0&0&0&0
\end{bmatrix}
\]
se representa la cadena $0\to1\to2\to3$, y la cerradura esperada es:
\[
T=
\begin{bmatrix}
0&1&1&1\\
0&0&1&1\\
0&0&0&1\\
0&0&0&0
\end{bmatrix}
\]

% =====================================================
\section{Representación matricial y costos}
\subsection{Memoria requerida}
Si se almacena cada celda como \texttt{uint8\_t} (1 byte), la memoria para la matriz es $N^2$ bytes.
La Tabla~\ref{tab:memoria} resume órdenes típicos.

\begin{table}[H]
\centering
\caption{Memoria aproximada para $A$ usando 1 byte por celda (sin contar overhead).}
\label{tab:memoria}
\begin{tabular}{@{}rrrr@{}}
\toprule
$N$ & $N^2$ & Memoria (bytes) & Aprox. \\ \midrule
1024 & 1{,}048{,}576 & 1{,}048{,}576 & $\sim$ 1 MB \\
2048 & 4{,}194{,}304 & 4{,}194{,}304 & $\sim$ 4 MB \\
4096 & 16{,}777{,}216 & 16{,}777{,}216 & $\sim$ 16 MB \\
8192 & 67{,}108{,}864 & 67{,}108{,}864 & $\sim$ 64 MB \\
\bottomrule
\end{tabular}
\end{table}

\subsection{Representación matricial de entrada y salida}
\textbf{Entrada.} La entrada es una matriz de adyacencia booleana $A\in\{0,1\}^{N\times N}$ donde cada celda codifica existencia de arco:
$A_{ij}=1$ si hay enlace directo $i\to j$, y $A_{ij}=0$ en caso contrario.

\textbf{Salida.} La salida es la \textbf{cerradura transitiva} $T\in\{0,1\}^{N\times N}$ (en esta implementación se obtiene \emph{in-place}
sobre $A$), donde:
\[
T_{ij}=1 \iff \exists \text{ un camino (longitud }\ge 1\text{) desde } i \text{ hasta } j.
\]
\textbf{Nota práctica:} dependiendo de la convención, puede incluirse o no la alcanzabilidad reflexiva ($T_{ii}=1$).
En este documento se usa la convención de caminos de longitud $\ge 1$ (por eso el ejemplo presenta diagonal cero).

\subsection{Arreglo lineal en memoria (fila-major) para CPU/GPU}
Para rendimiento y para facilitar copia a GPU, la matriz se almacena como un arreglo lineal contiguo (fila-major):
\[
A_{ij}\ \longleftrightarrow\ A[i\cdot N + j].
\]
Esta representación:
\begin{itemize}
  \item reduce \textit{overhead} de punteros (vs.\ \texttt{A[i][j]} con doble indirección),
  \item mejora localidad de caché en CPU,
  \item y en CUDA favorece accesos coalescentes cuando los hilos contiguos varían $j$.
\end{itemize}


\subsection{Costo computacional $O(N^3)$ en magnitudes reales}
El número de iteraciones del núcleo es $N^3$. Por ejemplo:
\[
1024^3 = 1{,}073{,}741{,}824,\qquad 2048^3 = 8{,}589{,}934{,}592,\qquad 4096^3 = 68{,}719{,}476{,}736.
\]
Esto explica por qué el algoritmo es ideal para evidenciar aceleración en GPU: la carga crece muy rápido con $N$.

\subsection{Determinación y justificación formal de la complejidad (Big O)}
El núcleo del algoritmo está compuesto por tres bucles anidados sobre $k$, $i$ y $j$, cada uno recorriendo $N$ elementos.
En el \textbf{peor caso} (matriz densa), el cuerpo interno ejecuta un número constante de operaciones booleanas
(lecturas, AND, OR y escritura), por lo que el costo total es:
\[
T(N)=\sum_{k=0}^{N-1}\sum_{i=0}^{N-1}\sum_{j=0}^{N-1} O(1)=N\cdot N\cdot N\cdot O(1)=O(N^3).
\]
Además, como el algoritmo necesariamente visita (en el peor caso) todas las combinaciones $(k,i,j)$,
también se cumple $T(N)=\Omega(N^3)$, y por tanto:
\[
T(N)=\Theta(N^3).
\]

\textbf{Efecto del atajo por $A_{ik}$.}
La optimización ``si $A_{ik}=0$ continuar'' puede reducir trabajo efectivo en matrices dispersas, pero \textbf{no cambia}
la cota asintótica del peor caso: si la matriz es densa (muchos $A_{ik}=1$), el algoritmo vuelve a ejecutar esencialmente
los $N^3$ pasos, manteniendo $O(N^3)$.

\textbf{Complejidad espacial.}
Se almacena una matriz booleana $N\times N$, por lo que el uso de memoria es:
\[
S(N)=O(N^2).
\]


\subsection{Dependencias de datos}
Existe una dependencia fuerte entre fases consecutivas $k$ y $k+1$:
todas las celdas $(i,j)$ de la fase $k$ deben completarse antes de pasar a $k+1$.
Sin embargo, para un $k$ fijo, las celdas $(i,j)$ se actualizan de manera independiente,
lo que habilita paralelismo masivo 2D.

% =====================================================
\section{Implementación secuencial (CPU) y mejoras básicas}
\subsection{Decisiones de implementación}
Para medir correctamente el rendimiento y preparar la Fase 2:
\begin{itemize}
  \item La matriz se almacena en un bloque contiguo de memoria (\texttt{uint8\_t*}), facilitando accesos y futura copia a GPU.
  \item Se usa semilla fija para reproducibilidad, y densidad $p$ para controlar lo disperso/denso del grafo.
  \item La medición cronometra \textbf{sólo el núcleo} (\texttt{k,i,j}), excluyendo generación e impresión.
\end{itemize}

\subsection{Optimización local (atajo por $A_{ik}$)}
Una mejora simple y segura: si $A_{ik}=0$, entonces $(A_{ik}\land A_{kj})=0$ para todo $j$,
por lo que la fila $i$ no cambia en esa fase $k$. Esto reduce trabajo cuando la matriz es dispersa.

\subsection{Código C secuencial (medición y verificación para casos pequeños)}
\begin{lstlisting}[language=c, caption={warshall\_menu.c (CPU) con medición del nucleo y verificacion opcional}]
// warshall_menu.c
// CPU secuencial: Cerradura transitiva booleana (Warshall logico)
// + MENU interactivo:
//   (1) ingresar matriz manual + parametros
//   (2) ingresar parametros + grafo random
//   (3) grafo y parametros random
//
// Mantiene modo clasico por argumentos:
//   ./warshall N p seed repeats verify [print]
//
// Compilar:
//   gcc -O3 -std=c11 warshall_menu.c -o warshall
// (si tu Linux requiere):
//   gcc -O3 -std=c11 warshall_menu.c -o warshall -lrt

#define _POSIX_C_SOURCE 200809L

#include <stdio.h>
#include <stdlib.h>
#include <stdint.h>
#include <string.h>
#include <time.h>
#include <errno.h>
#include <ctype.h>

static inline double seconds_now(void) {
    struct timespec ts;
#if defined(CLOCK_MONOTONIC)
    if (clock_gettime(CLOCK_MONOTONIC, &ts) != 0) {
        perror("clock_gettime");
        exit(EXIT_FAILURE);
    }
#else
    if (clock_gettime(CLOCK_REALTIME, &ts) != 0) {
        perror("clock_gettime");
        exit(EXIT_FAILURE);
    }
#endif
    return (double)ts.tv_sec + (double)ts.tv_nsec * 1e-9;
}

static uint8_t* alloc_matrix(int N) {
    size_t bytes = (size_t)N * (size_t)N * sizeof(uint8_t);
    uint8_t* A = (uint8_t*)malloc(bytes);
    if (!A) {
        fprintf(stderr, "ERROR: no se pudo asignar %zu bytes para N=%d\n", bytes, N);
        exit(EXIT_FAILURE);
    }
    return A;
}

static void init_random(uint8_t* A, int N, double p, unsigned seed) {
    srand(seed);
    for (int i = 0; i < N * N; i++) {
        double r = (double)rand() / (double)RAND_MAX;
        A[i] = (r < p) ? 1 : 0;
    }
}

static void print_matrix(const uint8_t* A, int N, const char* title) {
    printf("\n=== %s (N=%d) ===\n", title, N);

    printf("     ");
    for (int j = 0; j < N; j++) printf("%2d ", j);
    printf("\n");

    printf("     ");
    for (int j = 0; j < N; j++) printf("---");
    printf("\n");

    for (int i = 0; i < N; i++) {
        printf("%2d | ", i);
        const uint8_t* row = &A[i * N];
        for (int j = 0; j < N; j++) printf("%2d ", (int)row[j]);
        printf("\n");
    }
}

// -------------------------
// Nucleo: Warshall logico
// -------------------------
void warshall_logical(uint8_t* A, int N) {
    // A[i][j] = A[i][j] OR (A[i][k] AND A[k][j])
    // NO fuerza diagonal a 1.
    for (int k = 0; k < N; k++) {
        const uint8_t* row_k = &A[k * N];
        for (int i = 0; i < N; i++) {
            uint8_t aik = A[i * N + k];
            if (!aik) continue;
            uint8_t* row_i = &A[i * N];
            for (int j = 0; j < N; j++) {
                row_i[j] = (uint8_t)(row_i[j] | (aik & row_k[j]));
            }
        }
    }
}

// -----------------------------------
// Referencia: cerradura transitiva con BFS
// -----------------------------------
static void bfs_closure_ref(const uint8_t* Ain, uint8_t* Rref, int N) {
    int* queue = (int*)malloc((size_t)N * sizeof(int));
    uint8_t* vis = (uint8_t*)malloc((size_t)N * sizeof(uint8_t));
    if (!queue || !vis) {
        fprintf(stderr, "ERROR: memoria insuficiente para BFS\n");
        free(queue);
        free(vis);
        exit(EXIT_FAILURE);
    }

    for (int s = 0; s < N; s++) {
        memset(vis, 0, (size_t)N);
        int front = 0, back = 0;

        const uint8_t* row_s = &Ain[s * N];
        for (int v = 0; v < N; v++) {
            if (row_s[v]) {
                vis[v] = 1;
                queue[back++] = v;
            }
        }

        while (front < back) {
            int u = queue[front++];
            const uint8_t* row_u = &Ain[u * N];
            for (int v = 0; v < N; v++) {
                if (row_u[v] && !vis[v]) {
                    vis[v] = 1;
                    queue[back++] = v;
                }
            }
        }

        uint8_t* row_ref = &Rref[s * N];
        for (int j = 0; j < N; j++) row_ref[j] = vis[j];
    }

    free(queue);
    free(vis);
}

static int verify_against_ref(const uint8_t* Rref, const uint8_t* Aout, int N) {
    for (int i = 0; i < N; i++) {
        const uint8_t* rr = &Rref[i * N];
        const uint8_t* ao = &Aout[i * N];
        for (int j = 0; j < N; j++) {
            if (rr[j] != ao[j]) {
                fprintf(stderr,
                        "FALLO verificacion: fila i=%d, col j=%d | esperado=%d, obtenido=%d\n",
                        i, j, (int)rr[j], (int)ao[j]);
                return 0;
            }
        }
    }
    return 1;
}

// =====================================================
// Helpers de input robusto (sin scanf)
// =====================================================
static void read_line(char* buf, size_t n) {
    if (!fgets(buf, (int)n, stdin)) {
        printf("\nEOF detectado. Saliendo.\n");
        exit(0);
    }
}

static long read_long_prompt(const char* prompt, long minv, long maxv) {
    char line[256];
    for (;;) {
        printf("%s", prompt);
        read_line(line, sizeof(line));

        char* end = NULL;
        errno = 0;
        long v = strtol(line, &end, 10);
        if (errno == 0) {
            while (end && *end && isspace((unsigned char)*end)) end++;
            if (end && (*end == '\0' || *end == '\n')) {
                if (v >= minv && v <= maxv) return v;
            }
        }
        printf("Entrada invalida. Rango permitido: [%ld..%ld]\n", minv, maxv);
    }
}

static double read_double_prompt(const char* prompt, double minv, double maxv) {
    char line[256];
    for (;;) {
        printf("%s", prompt);
        read_line(line, sizeof(line));

        char* end = NULL;
        errno = 0;
        double v = strtod(line, &end);
        if (errno == 0) {
            while (end && *end && isspace((unsigned char)*end)) end++;
            if (end && (*end == '\0' || *end == '\n')) {
                if (v >= minv && v <= maxv) return v;
            }
        }
        printf("Entrada invalida. Rango permitido: [%.3f..%.3f]\n", minv, maxv);
    }
}

static int read_int_prompt(const char* prompt, int minv, int maxv) {
    return (int)read_long_prompt(prompt, (long)minv, (long)maxv);
}

static int read_yesno_prompt(const char* prompt) {
    char line[64];
    for (;;) {
        printf("%s (1=si, 0=no): ", prompt);
        read_line(line, sizeof(line));
        if (line[0] == '1') return 1;
        if (line[0] == '0') return 0;
        printf("Entrada invalida. Escribe 1 o 0.\n");
    }
}

static int parse_row_01(const char* line, uint8_t* row, int N) {
    // Acepta: "0 1 0 1" o "0101..." (con o sin espacios)
    int count = 0;
    for (const char* p = line; *p && count < N; p++) {
        if (*p == '0' || *p == '1') {
            row[count++] = (uint8_t)(*p - '0');
        }
    }
    return (count == N);
}

static double density_ones(const uint8_t* A, int N) {
    long long ones = 0;
    for (long long i = 0; i < (long long)N * (long long)N; i++) ones += A[i] ? 1 : 0;
    return (double)ones / (double)((long long)N * (long long)N);
}

// =====================================================
// Ejecutar en modo verify/timing (reusa tu logica)
// =====================================================
static int run_experiment(uint8_t* Ain, int N, double p, unsigned seed, int repeats, int verify, int print) {
    const int PRINT_LIMIT = 16;

    if (N <= PRINT_LIMIT) { // comportamiento original
        verify = 1;
        print = 1;
    }
    if (repeats <= 0) repeats = 1;

    uint8_t* A = alloc_matrix(N);

    if (verify) {
        if (N > 128) {
            printf("Aviso: verificacion activada con N=%d; se recomienda N<=128.\n", N);
        }

        memcpy(A, Ain, (size_t)N * (size_t)N);

        double t0 = seconds_now();
        warshall_logical(A, N);
        double t1 = seconds_now();
        double kernel_time = t1 - t0;

        uint8_t* Rref = alloc_matrix(N);
        bfs_closure_ref(Ain, Rref, N);

        int ok = verify_against_ref(Rref, A, N);

        if (print) {
            print_matrix(Ain,  N, "MATRIZ DE ENTRADA (Grafo / Adyacencia)");
            print_matrix(Rref, N, "MATRIZ DE VERIFICACION (Referencia BFS)");
            print_matrix(A,    N, "MATRIZ DE SALIDA (Warshall logico)");
        }

        printf("\nVALIDACION (BFS) para N=%d: %s\n", N, ok ? "OK" : "FALLIDA");
        printf("Tiempo del nucleo (warshall_logical): %.6f s\n", kernel_time);
        printf("Resumen params | N=%d | p=%.3f | seed=%u | repeats=%d | verify=%d | print=%d\n",
               N, p, seed, repeats, verify, print);

        free(Rref);
        free(A);
        return ok ? EXIT_SUCCESS : EXIT_FAILURE;
    }

    double best = 1e100;
    for (int r = 0; r < repeats; r++) {
        memcpy(A, Ain, (size_t)N * (size_t)N);
        double t0 = seconds_now();
        warshall_logical(A, N);
        double t1 = seconds_now();
        double dt = t1 - t0;
        if (dt < best) best = dt;
    }

    printf("CPU Warshall logico | N=%d | p=%.3f | seed=%u | repeats=%d | best_kernel_time=%.6f s\n",
           N, p, seed, repeats, best);

    free(A);
    return EXIT_SUCCESS;
}

// =====================================================
// Menu
// =====================================================
static void menu_loop(void) {
    for (;;) {
        printf("\n==============================\n");
        printf("   MENU - Warshall logico CPU\n");
        printf("==============================\n");
        printf("1) Ingresar MATRIZ manual + parametros\n");
        printf("2) Ingresar parametros + GRAFO random\n");
        printf("3) Grafo y parametros RANDOM\n");
        printf("0) Salir\n");

        int opt = read_int_prompt("Opcion: ", 0, 3);
        if (opt == 0) break;

        int N = 256;
        double p = 0.05;
        unsigned seed = 1234;
        int repeats = 3;
        int verify = 0;
        int print = 0;

        uint8_t* Ain = NULL;

        if (opt == 1) {
            N = read_int_prompt("Ingrese N (1..2048 recomendado): ", 1, 4096);
            Ain = alloc_matrix(N);

            printf("\nIngrese la matriz de adyacencia (%dx%d) con 0/1.\n", N, N);
            printf("Formato permitido por fila: '0 1 0 1' o '0101...'\n\n");

            char line[8192];
            for (int i = 0; i < N; i++) {
                for (;;) {
                    printf("Fila %d: ", i);
                    read_line(line, sizeof(line));
                    if (parse_row_01(line, &Ain[i * N], N)) break;
                    printf("Fila invalida. Debe contener %d valores 0/1.\n", N);
                }
            }

            // p se calcula como densidad real (informativo)
            p = density_ones(Ain, N);
            seed = 0;

            repeats = read_int_prompt("repeats (>=1): ", 1, 1000);
            verify  = read_yesno_prompt("verify");
            print   = read_yesno_prompt("print");

            printf("\nDensidad p calculada desde la matriz: %.3f\n", p);
        }
        else if (opt == 2) {
            N = read_int_prompt("N (1..4096): ", 1, 4096);
            p = read_double_prompt("p (0..1): ", 0.0, 1.0);
            seed = (unsigned)read_long_prompt("seed (0..2^31-1): ", 0, 2147483647L);
            repeats = read_int_prompt("repeats (>=1): ", 1, 1000);
            verify  = read_yesno_prompt("verify");
            print   = read_yesno_prompt("print");

            Ain = alloc_matrix(N);
            init_random(Ain, N, p, seed);
        }
        else if (opt == 3) {
            // Random razonable (puedes ajustar los rangos)
            unsigned s = (unsigned)time(NULL);
            srand(s);

            int choices[] = {8,16,32,64,128,256,512,1024};
            int idx = rand() % (int)(sizeof(choices)/sizeof(choices[0]));
            N = choices[idx];

            p = 0.01 + ((double)rand() / (double)RAND_MAX) * 0.19; // [0.01..0.20]
            seed = (unsigned)rand();
            repeats = 1 + (rand() % 7); // 1..7

            // si N tiene un size chico, validamos; si no, solo timing
            verify = (N <= 128) ? 1 : 0;
            print  = (N <= 16) ? 1 : 0;

            Ain = alloc_matrix(N);
            init_random(Ain, N, p, seed);

            printf("\nParametros random generados:\n");
            printf("N=%d | p=%.3f | seed=%u | repeats=%d | verify=%d | print=%d\n",
                   N, p, seed, repeats, verify, print);
        }

        int rc = run_experiment(Ain, N, p, seed, repeats, verify, print);
        free(Ain);

        if (rc != EXIT_SUCCESS) {
            printf("Ejecucion termino con error (validacion fallida o problema).\n");
        }

        if (!read_yesno_prompt("\nDeseas ejecutar otra vez?")) break;
    }
}

int main(int argc, char** argv) {
    // --- Modo por argumentos (tu modo original) ---
    if (argc >= 2) {
        int N = 256;
        double p = 0.05;
        unsigned seed = 1234;
        int repeats = 3;
        int verify = 0;
        int print = 0;

        if (argc >= 2) N = atoi(argv[1]);
        if (argc >= 3) p = atof(argv[2]);
        if (argc >= 4) seed = (unsigned)atoi(argv[3]);
        if (argc >= 5) repeats = atoi(argv[4]);
        if (argc >= 6) verify = atoi(argv[5]);
        if (argc >= 7) print = atoi(argv[6]);

        if (N <= 0) { fprintf(stderr, "ERROR: N debe ser > 0\n"); return EXIT_FAILURE; }
        if (p < 0.0 || p > 1.0) { fprintf(stderr, "ERROR: p debe estar en [0,1]\n"); return EXIT_FAILURE; }
        if (repeats <= 0) repeats = 1;

        const int PRINT_LIMIT = 16;
        if (N <= PRINT_LIMIT) { verify = 1; print = 1; }

        uint8_t* Ain = alloc_matrix(N);
        init_random(Ain, N, p, seed);

        int rc = run_experiment(Ain, N, p, seed, repeats, verify, print);
        free(Ain);
        return rc;
    }

    // --- Modo menu ---
    menu_loop();
    return 0;
}
\end{lstlisting}

Como se puede apreciar, el código incluye una función de verificación, que comprueba la salida de nuestro código usando BFS (búsqueda en anchura) para determinar si hay o no camino los vértices que se indican en la salida de nuestro algoritmo.

% =====================================================
\section{Metodología de pruebas y análisis experimental}
\subsection{Variables controladas}
Para asegurar comparabilidad:
\begin{itemize}
  \item \textbf{Semilla fija}: garantiza mismos datos en ejecuciones repetidas.
  \item \textbf{Densidad $p$}: controla la probabilidad de 1 (matriz dispersa vs densa).
  \item \textbf{Medición del núcleo}: sólo el tiempo del triple bucle (excluye generación e I/O).
\end{itemize}

\subsection{Tamaños de prueba}
Se recomienda ejecutar:
\[
N \in \{128, 256, 512, 1024, 2048\}
\]
(extendible a 4096 si la memoria/tiempo lo permite). Para esta fase, lo crítico es incluir $N\ge 1024$.

\subsection{Compilación y ejecución}
\begin{verbatim}
# Compilar (Linux)
gcc -O3 -std=c11 warshall_menu.c -o warshall

# Ejecutar: ./warshall N p seed repeats verify print
./warshall 256  0.05 1234 5 0 0
./warshall 1024 0.02 1234 3 0 0
./warshall 2048 0.02 1234 3 0 0
./warshall 4096 0.005 1234 2 0 0
\end{verbatim}

\subsection{Plantilla de resultados (CPU secuencial)}
\begin{table}[H]
\centering
\caption{Resultados de tiempo (CPU secuencial).}
\label{tab:cpu}
\begin{tabular}{@{}rrrrr@{}}
\toprule
$N$ & $N^2$ & $N^3$ & $p$ & Tiempo núcleo (s) \\ \midrule
128  & 16{,}384    & 2{,}097{,}152      & 0.05 & 0.000062 \\
256  & 65{,}536    & 16{,}777{,}216     & 0.05 & 0.000808 \\
512  & 262{,}144   & 134{,}217{,}728    & 0.02 & 0.008243 \\
1024 & 1{,}048{,}576 & 1{,}073{,}741{,}824 & 0.02 & 0.075686 \\
2048 & 4{,}194{,}304 & 8{,}589{,}934{,}592 & 0.02 & 0.630069 \\
4096 & 16{,}777{,}216 & 6{,}719{,}476{,}736 & 0.02 & 5.269098 \\
\bottomrule
\end{tabular}
\end{table}

En la siguiente imagen se evidencia el crecimiento cúbico del tiempo con respecto a $N$.
\begin{figure}[H]
\centering
\includegraphics[width=0.7\textwidth]{imagenes/warshall.jpeg}
\caption{Tiempo del núcleo vs. $N$ (CPU secuencial).}
\label{fig:cpu_tiempo}
\end{figure}

\subsection{Discusión de resultados}
\begin{itemize}
  \item Al duplicar $N$, el tiempo tiende a crecer aproximadamente por un factor cercano a 8 (por $N^3$).
  \item Para matrices dispersas (p pequeño), el atajo por $A_{ik}$ puede reducir significativamente el trabajo efectivo.
  \item Para matrices densas (p grande), la mejora por el atajo disminuye y el costo se acerca más al cúbico completo.
\end{itemize}

% =====================================================
\section{Conclusiones}
\begin{itemize}
  \item Warshall lógico computa cerradura transitiva booleana sobre una matriz $N\times N$ y resuelve alcanzabilidad all-pairs.
  \item Su costo temporal \textbf{$O(N^3)$} y su estructura matricial lo convierten en un candidato fuerte para aceleración con CUDA
  cuando $N\ge 1024$.
  \item La dependencia por fases en $k$ obliga a sincronizar entre iteraciones, pero dentro de cada fase existe paralelismo masivo 2D.
  \item La implementación secuencial propuesta es reproducible y medible (semilla fija, densidad $p$, medición del núcleo),
  permitiendo comparar con la futura versión CUDA de manera confiable.
  \item Para Fase 2, el diseño recomendado es un kernel por $k$ con grilla 2D, cuidando coalescencia y (opcionalmente) \textit{tiling}
  para reducir accesos a memoria global.
\end{itemize}

% =====================================================


% =====================================================
%% BIBLIOGRAFÍA
\printreferences[Referencias Bibliográficas]

\end{document}
