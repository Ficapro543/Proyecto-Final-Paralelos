\documentclass[10pt,a4paper]{article}  % 12pt según APA
%% Carga del paquete de formato personalizado
\usepackage{formato}

%% =====================================================
%% CONFIGURACIÓN DE LOS DATOS DEL DOCUMENTO
%% =====================================================
\documenttype{Proyecto N° 1}
\documenttitle{Algoritmos de Ordenamiento Paralelos y Secuenciales}
\subject{Algoritmos Paralelos y Distribuidos}
\advisor{
Mgt. Dueñas Jimenez, Ray}
\semester{2025-II}

%% =====================================================
%% REGISTRO DE AUTORES/ESTUDIANTES
%% =====================================================
% Para múltiples autores
\addauthor{Castro Pari, Rayneld Fidel}
\addauthor{Mayhuire Chacon, Brenda Lucia}
\addauthor{Mendoza Quispe, Jose Daniel}
\addauthor{Perez Cahuana, Gabriel}
\addauthor{Zevallos Yanqui, Andy Jefferson}
%\addauthor{Apellido1 Apellido2, Nombre1 Nombre2}
%\addauthor{Apellido1 Apellido2, Nombre1 Nombre2}

%% =====================================================
%% INICIO DEL DOCUMENTO
%% =====================================================
\begin{document}

%% =====================================================
%% PORTADA
%% =====================================================
\thispagestyle{empty}  % La portada no tiene numeración de página
\makeatletter
%% =====================================================
%% Portada - Documento académico - Versión APA
%% =====================================================
\begin{titlepage}
    \thispagestyle{empty}
    
    % Cambiar a interlineado sencillo solo para la portada
    \singlespacing
    
    %% =====================================================
    %% ENCABEZADO INSTITUCIONAL
    %% =====================================================
    \begin{center}
        UNIVERSIDAD NACIONAL DE SAN ANTONIO ABAD DEL CUSCO\\[0.3cm]
        FACULTAD DE INGENIERÍA ELÉCTRICA, ELECTRÓNICA, INFORMÁTICA Y MECÁNICA\\[0.3cm]
        ESCUELA PROFESIONAL DE INGENIERÍA INFORMÁTICA Y DE SISTEMAS
    \end{center}

    
    %% =====================================================
    %% ESCUDO/LOGO INSTITUCIONAL
    %% =====================================================
    \vspace{0.25cm}
    \begin{figure}[H]
        \centering
        \includegraphics[scale=0.3]{imagenes/escudo.png}
    \end{figure}
    
    %% =====================================================
    %% TÍTULO DEL DOCUMENTO
    %% =====================================================
    \begin{center}
        \Large{\textbf{\@documenttype}} \\[0.3cm]
        \rule{160mm}{0.6mm} \\[0.3cm]
        {\Large \textbf{\@documenttitle}} \\[0.1cm]
        \rule{160mm}{0.6mm}
    \end{center}
    
    \vspace{1cm}
    
    %% =====================================================
    %% INFORMACIÓN DEL CURSO Y PARTICIPANTES
    %% =====================================================
    \noindent\hspace*{2cm}%
    \parbox{0.5\textwidth}{
        \raggedright
        \textbf{Asignatura:} \\
        \hspace*{2em} \@subject \\[0.8cm]
        \textbf{Docente:} \\
        \hspace*{2em} \@advisor \\[0.8cm]
        % Condicional para singular/plural de "Estudiante(s)"
        \textbf{\ifnum\value{authorcount}>1 Estudiantes:\else Estudiante:\fi} \\
        \hspace*{2em}
        \parbox{\dimexpr\linewidth-2em\relax}{
            \raggedright
            \vspace{0.3em}
            \authorlist
        }
    }
    

    
    \vfill
    
    %% =====================================================
    %% FECHA DE PUBLICACIÓN
    %% =====================================================
    \begin{center}
        Cusco - Perú \\ \the\year
    \end{center}
    
    % Restaurar interlineado doble para el resto del documento
    \doublespacing
\end{titlepage}    % Incluye la portada
\makeatother

%% =====================================================
%% INICIO DE SECCIONES PRELIMINARES CON NUMERACIÓN ROMANA
%% =====================================================
\startpreliminary
\clearpage

%% =====================================================
%% TABLA DE CONTENIDOS
%% =====================================================
\tableofcontents
\clearpage

%% =====================================================
%% ÍNDICES DE FIGURAS Y TABLAS (Opcional)
%% =====================================================
% Descomenta si tienes figuras y tablas
\listoffigures
\clearpage
%\listoftables
%\clearpage

%% =====================================================
%% CAMBIO A NUMERACIÓN ARÁBIGA PARA EL CONTENIDO PRINCIPAL
%% =====================================================
\startmain

%% =====================================================
%% CONTENIDO PRINCIPAL DEL DOCUMENTO
%% =====================================================
%================================================
\section*{1.\ \ \ Resumen}
\addcontentsline{toc}{section}{1.\ \ \ Resumen}

La \textbf{cerradura transitiva booleana} (también conocida como \textbf{Warshall lógico}) es un algoritmo clásico de álgebra booleana
y teoría de grafos que permite determinar la \textbf{alcanzabilidad} entre todos los pares de vértices de un grafo dirigido.
A partir de una matriz de adyacencia $A \in \{0,1\}^{N \times N}$, el algoritmo produce una matriz $T$ donde $T_{ij}=1$ si existe
un camino (de longitud $\ge 0$ o $\ge 1$ según convención) desde $i$ hasta $j$.

Este trabajo propone Warshall lógico como candidato para la \textbf{paralelización con CUDA} debido a su estructura matricial,
su costo computacional \textbf{$O(N^3)$} y su verificación directa comparando contra la versión secuencial. Se describe el algoritmo,
su representación matricial, su complejidad y un diseño de implementación secuencial en C/C++ con medición de tiempo para diferentes
tamaños de $N$ (incluyendo $N \ge 1024$) y verificación para matrices pequeñas.

\clearpage

%================================================
\section*{2.\ \ \ Introducción}
\addcontentsline{toc}{section}{2.\ \ \ Introducción}

En problemas reales de computación científica e ingeniería, es frecuente trabajar con \textbf{matrices masivas}:
modelos de conectividad, relaciones de dependencia, grafos de estados, análisis de flujos, rutas, etc.
Cuando el tamaño $N$ crece (por ejemplo $N \ge 1024$), el costo de los algoritmos cúbicos puede volverse prohibitivo
en CPU, lo cual motiva el uso de paralelización en GPU (CUDA).

El algoritmo de Warshall lógico pertenece a la familia de algoritmos \textbf{All-Pairs Reachability} y es una variante booleana
del cálculo de caminos. Su núcleo consiste en actualizar $A_{ij}$ usando operaciones booleanas con un índice intermedio $k$,
manteniendo una dependencia por iteración de $k$ que es relevante para el diseño paralelo.

\clearpage

%================================================
\section*{3.\ \ \ Objetivo General}
\addcontentsline{toc}{section}{3.\ \ \ Objetivo General}

\textbf{Diseñar, implementar y evaluar} la aceleración obtenida al paralelizar el algoritmo de \textbf{cerradura transitiva booleana}
(Warshall lógico) para matrices cuadradas masivas ($N \ge 1024$), comparando:
\begin{itemize}
  \item Versión secuencial (CPU).
  \item Versión paralela (CUDA) en Fase 2.
  \item (Opcional en Fase 2) Versión CPU multi-core (OpenMP) para comparación.
\end{itemize}

\clearpage

%================================================
\section*{4.\ \ \ Warshall Lógico: Versión Secuencial}
\addcontentsline{toc}{section}{4.\ \ \ Warshall Lógico: Versión Secuencial}

El algoritmo de Warshall lógico calcula la \textbf{cerradura transitiva} de un grafo dirigido usando su matriz de adyacencia.
Sea $A$ la matriz de entrada, donde:
\[
A_{ij} =
\begin{cases}
1, & \text{si existe un arco directo } i \to j\\
0, & \text{caso contrario}
\end{cases}
\]
La idea es permitir, progresivamente, que los caminos utilicen vértices intermedios del conjunto $\{0,1,\dots,k\}$.

\subsection*{4.1.\ \ Requisitos del Algoritmo}
\addcontentsline{toc}{subsection}{4.1.\ \ Requisitos del Algoritmo}

\textbf{(a) Operación con matrices masivas:} El algoritmo trabaja con una matriz cuadrada booleana $N \times N$.
Para el proyecto se requiere $N \ge 1024$.

\textbf{(b) Complejidad mínima:} Warshall lógico presenta complejidad temporal:
\[
T(N) = O(N^3),
\]
lo cual es un excelente candidato para paralelización porque el costo crece rápidamente con $N$.

\textbf{(c) Verificabilidad:} La versión paralela puede validarse comparando su salida con la versión secuencial (CPU) usando:
\begin{itemize}
  \item Comparación elemento a elemento para matrices pequeñas (por ejemplo $N \le 128$).
  \item Pruebas aleatorias (random testing) con semillas fijas.
  \item Casos controlados (grafos con estructura conocida: cadena, ciclo, completo, vacío).
\end{itemize}

\clearpage

%================================================
\section*{5.\ \ \ Descripción del Algoritmo}
\addcontentsline{toc}{section}{5.\ \ \ Descripción del Algoritmo}

\subsection*{5.1.\ \ Campo de aplicación}
\addcontentsline{toc}{subsection}{5.1.\ \ Campo de aplicación}

La cerradura transitiva se usa en:
\begin{itemize}
  \item \textbf{Teoría de grafos}: alcanzabilidad, componentes, análisis de conectividad.
  \item \textbf{Bases de datos}: consultas de tipo \textit{reachability} sobre relaciones (p.ej., jerarquías, dependencias).
  \item \textbf{Compiladores y análisis de programas}: grafos de flujo, dependencias entre módulos.
  \item \textbf{Redes y sistemas}: análisis de rutas y acceso entre nodos.
\end{itemize}

\subsection*{5.2.\ \ Descripción detallada del funcionamiento}
\addcontentsline{toc}{subsection}{5.2.\ \ Descripción detallada del funcionamiento}

Warshall lógico aplica la recurrencia:
\[
A_{ij} \leftarrow A_{ij}\ \lor\ (A_{ik} \land A_{kj})
\quad \text{para } k=0,\dots,N-1.
\]
Intuición:
\begin{itemize}
  \item $A_{ik}=1$ indica que $i$ puede llegar a $k$.
  \item $A_{kj}=1$ indica que $k$ puede llegar a $j$.
  \item Si ambos son verdaderos, entonces $i$ puede llegar a $j$ usando a $k$ como intermedio.
\end{itemize}

\subsubsection*{Pseudocódigo (Warshall lógico)}
\addcontentsline{toc}{subsubsection}{Pseudocódigo (Warshall lógico)}

\begin{algorithm}[H]
\caption{Cerradura transitiva booleana (Warshall lógico)}
\KwIn{Matriz booleana $A \in \{0,1\}^{N\times N}$}
\KwOut{Matriz booleana $A$ actualizada con su cerradura transitiva}
\For{$k \leftarrow 0$ \KwTo $N-1$}{
  \For{$i \leftarrow 0$ \KwTo $N-1$}{
    \For{$j \leftarrow 0$ \KwTo $N-1$}{
      $A[i][j] \leftarrow A[i][j]\ \lor\ (A[i][k] \land A[k][j])$\;
    }
  }
}
\end{algorithm}

\subsubsection*{Ejemplo pequeño}
Considere $N=4$ y la matriz de adyacencia:
\[
A=
\begin{bmatrix}
0&1&0&0\\
0&0&1&0\\
0&0&0&1\\
0&0&0&0
\end{bmatrix}
\]
Esto representa la cadena $0\to1\to2\to3$. La cerradura transitiva resultante debe permitir:
$0$ llega a $2$ y $3$, y $1$ llega a $3$. Por tanto:
\[
T=
\begin{bmatrix}
0&1&1&1\\
0&0&1&1\\
0&0&0&1\\
0&0&0&0
\end{bmatrix}
\]

\clearpage

%================================================
\section*{6.\ \ \ Representación Matricial}
\addcontentsline{toc}{section}{6.\ \ \ Representación Matricial}

\subsection*{6.1.\ \ Representación de los datos de entrada}
\addcontentsline{toc}{subsection}{6.1.\ \ Representación de los datos de entrada}

La entrada se modela como una matriz booleana $A$:
\[
A \in \{0,1\}^{N\times N}.
\]
Cada fila $i$ representa los destinos alcanzables desde $i$ en un paso (aristas directas).

\textbf{Memoria:} almacenar $A$ como \texttt{uint8\_t} (1 byte) requiere $N^2$ bytes.
Para $N=1024$, $N^2=1{,}048{,}576$ entradas $\approx 1$ MB.

\subsection*{6.2.\ \ Representación de la operación núcleo}
\addcontentsline{toc}{subsection}{6.2.\ \ Representación de la operación núcleo}

La actualización elemento a elemento es:
\[
A_{ij}^{(k)} = A_{ij}^{(k-1)} \lor \left(A_{ik}^{(k-1)} \land A_{kj}^{(k-1)}\right).
\]
Donde el superíndice $(k)$ indica que se permiten vértices intermedios hasta $k$.

\subsection*{6.3.\ \ Representación de los datos de salida}
\addcontentsline{toc}{subsection}{6.3.\ \ Representación de los datos de salida}

La salida es la misma matriz $A$ ya actualizada (in-place) o una matriz $T$ separada.
En ambos casos:
\[
T_{ij}=1 \iff \exists \text{ un camino de } i \text{ a } j.
\]

\clearpage

%================================================
\section*{7.\ \ \ Complejidad Computacional}
\addcontentsline{toc}{section}{7.\ \ \ Complejidad Computacional}

\subsection*{7.1.\ \ Parámetros relevantes}
\addcontentsline{toc}{subsection}{7.1.\ \ Parámetros relevantes}

\begin{itemize}
  \item $N$: dimensión de la matriz cuadrada ($N \times N$).
  \item Operación booleana básica: $\lor$ y $\land$ (tiempo constante en CPU).
  \item Accesos a memoria: lecturas de $A[i][j]$, $A[i][k]$, $A[k][j]$ y escritura de $A[i][j]$.
\end{itemize}

\subsubsection*{7.1.1.\ \ Complejidad temporal}
\addcontentsline{toc}{subsubsection}{7.1.1.\ \ Complejidad temporal}

El algoritmo ejecuta tres bucles anidados de tamaño $N$, por lo tanto:
\[
T(N)=\sum_{k=1}^{N}\sum_{i=1}^{N}\sum_{j=1}^{N} O(1)=O(N^3).
\]

\subsubsection*{7.1.2.\ \ Complejidad espacial}
\addcontentsline{toc}{subsubsection}{7.1.2.\ \ Complejidad espacial}

La memoria principal está dominada por la matriz:
\[
S(N)=O(N^2).
\]

\subsubsection*{7.1.3.\ \ Implicancias para el rendimiento}
\addcontentsline{toc}{subsubsection}{7.1.3.\ \ Implicancias para el rendimiento}

Para $N$ grande, el tiempo está fuertemente influenciado por:
\begin{itemize}
  \item \textbf{Localidad de memoria}: el acceso repetido a filas/columnas puede causar fallos de caché.
  \item \textbf{Ancho de banda}: $O(N^3)$ operaciones implican múltiples lecturas/escrituras.
  \item \textbf{Dependencias}: existe una dependencia fuerte por iteración de $k$, lo cual define el diseño paralelo.
\end{itemize}

\clearpage

%================================================
\section*{8.\ \ \ Análisis del Algoritmo}
\addcontentsline{toc}{section}{8.\ \ \ Análisis del Algoritmo}

\subsection*{8.1.\ \ Funcionamiento del Algoritmo}
\addcontentsline{toc}{subsection}{8.1.\ \ Funcionamiento del Algoritmo}

La iteración externa ($k$) se interpreta como: ``permitir que $k$ sea vértice intermedio''.
Para cada $k$, se actualizan todos los pares $(i,j)$ en la matriz.

\textbf{Dependencia de datos clave:}
\begin{itemize}
  \item Para un $k$ fijo, cada celda $(i,j)$ se puede calcular de manera independiente \textbf{si se leen} $A[i][k]$ y $A[k][j]$
  consistentes con la iteración $k$.
  \item Sin embargo, pasar de $k$ a $k+1$ requiere que todas las actualizaciones de $k$ estén completadas.
\end{itemize}

Esto sugiere un patrón típico para CUDA:
\begin{itemize}
  \item Mantener el bucle $k$ en CPU o en GPU de forma secuencial.
  \item Para cada $k$, paralelizar la actualización de $A[i][j]$ sobre una grilla 2D de hilos.
\end{itemize}

\subsection*{8.2.\ \ Justificación del Algoritmo como candidato CUDA}
\addcontentsline{toc}{subsection}{8.2.\ \ Justificación del Algoritmo}

Warshall lógico cumple los requisitos del proyecto:
\begin{itemize}
  \item \textbf{Matrices masivas:} $A$ es $N \times N$ y puede evaluarse con $N\ge 1024$.
  \item \textbf{Alta complejidad:} $O(N^3)$ (categoría ``Alto'').
  \item \textbf{Paralelizable:} para un $k$ fijo, las celdas $(i,j)$ son paralelizables.
  \item \textbf{Verificación directa:} comparación bit a bit contra la versión secuencial.
\end{itemize}

\textbf{Nota de optimización para Fase 2 (opcional):} En GPU suele ser ventajoso aplicar una versión \textbf{bloqueada (tiled)}
para aprovechar memoria compartida y mejorar coalescencia, similar a Floyd-Warshall bloqueado.

\subsection*{8.3.\ \ Ejecución y Medición del Tiempo (CPU)}
\addcontentsline{toc}{subsection}{8.3.\ \ Ejecución y Medición del Tiempo (CPU)}

La medición debe contabilizar únicamente el \textbf{núcleo del algoritmo} (los bucles $k,i,j$), excluyendo:
\begin{itemize}
  \item Generación/lectura de la matriz.
  \item Impresión de matrices.
  \item Verificación (cuando se mida rendimiento).
\end{itemize}

Se recomienda usar \texttt{clock\_gettime(CLOCK\_MONOTONIC, ...)} para medir en segundos.

\subsection*{8.4.\ \ Medición del rendimiento (diferentes tamaños)}
\addcontentsline{toc}{subsection}{8.4.\ \ Medición del rendimiento (diferentes tamaños)}

La ejecución debe repetirse para varios tamaños:
\[
N \in \{128,256,512,1024,2048,\dots\}
\]
y registrar el tiempo $t(N)$.

\begin{table}[H]
\centering
\caption{Plantilla de resultados (CPU secuencial)}
\begin{tabular}{@{}rrrr@{}}
\toprule
$N$ & Entradas ($N^2$) & Tiempo (s) & Observaciones \\ \midrule
128  & 16384   & -- & Verificación habilitada \\
256  & 65536   & -- & -- \\
512  & 262144  & -- & -- \\
1024 & 1048576 & -- & Caso masivo requerido \\
\bottomrule
\end{tabular}
\end{table}

\clearpage

%================================================
\section*{9.\ \ \ Conclusiones}
\addcontentsline{toc}{section}{9.\ \ \ Conclusiones}

\begin{itemize}
  \item Warshall lógico permite calcular la cerradura transitiva booleana de una matriz de adyacencia y resolver alcanzabilidad
  all-pairs en grafos dirigidos.
  \item Su complejidad \textbf{$O(N^3)$} lo vuelve un candidato fuerte para acelerar con CUDA cuando $N \ge 1024$.
  \item La dependencia por iteración $k$ impone un esquema paralelo por ``fases'' (un kernel por $k$ o un enfoque bloqueado),
  pero dentro de cada $k$ el paralelismo sobre $(i,j)$ es masivo.
  \item La validación es clara: la salida paralela debe coincidir con la salida secuencial.
\end{itemize}

\clearpage

%================================================
\section*{10.\ \ \ ANEXOS}
\addcontentsline{toc}{section}{10.\ \ \ ANEXOS}

\subsection*{A. Implementación secuencial en C (con medición de tiempo y verificación)}
\addcontentsline{toc}{subsection}{A. Implementación secuencial en C (con medición de tiempo y verificación)}

\textbf{Estructura propuesta:}
\begin{itemize}
  \item \texttt{alloc\_matrix}: reserva memoria contigua $N^2$.
  \item \texttt{init\_random}: genera matriz booleana (densidad controlable).
  \item \texttt{warshall\_logical}: núcleo $O(N^3)$.
  \item \texttt{verify\_small}: comparación con versión alternativa o pruebas controladas.
  \item \texttt{timing}: medición del tiempo sólo del núcleo.
\end{itemize}

\begin{lstlisting}[caption={warshall_logico_secuencial.c}]
#include <stdio.h>
#include <stdlib.h>
#include <stdint.h>
#include <time.h>

static inline double seconds_now() {
    struct timespec ts;
    clock_gettime(CLOCK_MONOTONIC, &ts);
    return (double)ts.tv_sec + (double)ts.tv_nsec * 1e-9;
}

uint8_t* alloc_matrix(int N) {
    uint8_t* A = (uint8_t*)malloc((size_t)N * (size_t)N * sizeof(uint8_t));
    if (!A) {
        fprintf(stderr, "Error: no se pudo asignar memoria para N=%d\n", N);
        exit(EXIT_FAILURE);
    }
    return A;
}

void init_random(uint8_t* A, int N, double p, unsigned seed) {
    // p = probabilidad de 1 (densidad). Ej: p=0.05 -> matriz dispersa
    srand(seed);
    for (int i = 0; i < N*N; i++) {
        double r = (double)rand() / (double)RAND_MAX;
        A[i] = (r < p) ? 1 : 0;
    }
}

void init_chain_example(uint8_t* A, int N) {
    // Ejemplo controlado: cadena 0->1->2->...->N-1
    for (int i = 0; i < N*N; i++) A[i] = 0;
    for (int i = 0; i < N-1; i++) A[i*N + (i+1)] = 1;
}

void print_matrix(const uint8_t* A, int N) {
    for (int i = 0; i < N; i++) {
        for (int j = 0; j < N; j++) {
            printf("%d ", (int)A[i*N + j]);
        }
        printf("\n");
    }
}

void warshall_logical(uint8_t* A, int N) {
    // A[i][j] = A[i][j] OR (A[i][k] AND A[k][j])
    for (int k = 0; k < N; k++) {
        for (int i = 0; i < N; i++) {
            uint8_t aik = A[i*N + k];
            if (!aik) continue; // pequeño ahorro: si A[i][k]==0, no aporta
            for (int j = 0; j < N; j++) {
                A[i*N + j] = (uint8_t)(A[i*N + j] | (aik & A[k*N + j]));
            }
        }
    }
}

int matrices_equal(const uint8_t* A, const uint8_t* B, int N) {
    for (int i = 0; i < N*N; i++) {
        if (A[i] != B[i]) return 0;
    }
    return 1;
}

int main(int argc, char** argv) {
    int N = 256;
    double p = 0.05;
    unsigned seed = 1234;
    int verify = 0;

    if (argc >= 2) N = atoi(argv[1]);        // ./a.out N
    if (argc >= 3) p = atof(argv[2]);        // ./a.out N p
    if (argc >= 4) seed = (unsigned)atoi(argv[3]); // ./a.out N p seed
    if (argc >= 5) verify = atoi(argv[4]);   // ./a.out N p seed verify

    uint8_t* A = alloc_matrix(N);
    init_random(A, N, p, seed);

    // Verificación sólo para N pequeño (opcional)
    // Aquí se muestra un ejemplo de verificación con un caso controlado "cadena"
    if (verify && N <= 64) {
        uint8_t* Atest = alloc_matrix(N);
        uint8_t* Aref  = alloc_matrix(N);
        init_chain_example(Atest, N);
        for (int i = 0; i < N*N; i++) Aref[i] = Atest[i];

        warshall_logical(Atest, N);

        // Propiedad esperada para cadena: si i<j entonces alcanzable (1) para j>i
        for (int i = 0; i < N; i++) {
            for (int j = i+1; j < N; j++) {
                Aref[i*N + j] = 1;
            }
        }

        if (!matrices_equal(Atest, Aref, N)) {
            fprintf(stderr, "Verificacion FALLIDA para N=%d\n", N);
            fprintf(stderr, "Salida obtenida:\n"); print_matrix(Atest, N);
            fprintf(stderr, "Salida esperada:\n"); print_matrix(Aref, N);
            return EXIT_FAILURE;
        } else {
            printf("Verificacion OK para N=%d (caso cadena)\n", N);
        }
        free(Atest); free(Aref);
    }

    // Medición del núcleo
    double t0 = seconds_now();
    warshall_logical(A, N);
    double t1 = seconds_now();

    printf("N=%d, p=%.3f, tiempo_nucleo=%.6f s\n", N, p, (t1 - t0));

    free(A);
    return 0;
}
\end{lstlisting}

\subsection*{B. Comandos de compilación y ejecución (Linux/Colab)}
\addcontentsline{toc}{subsection}{B. Comandos de compilación y ejecución (Linux/Colab)}

\begin{verbatim}
# Compilar (GCC)
gcc -O3 -std=c11 warshall_logico_secuencial.c -o warshall

# Ejecutar: ./warshall N p seed verify
# N: tamaño, p: densidad, seed: semilla, verify: 1/0
./warshall 256 0.05 1234 1
./warshall 1024 0.02 1234 0
\end{verbatim}

\subsection*{C. Cómo se evaluará en Fase 2 (referencia del proyecto)}
\addcontentsline{toc}{subsection}{C. Cómo se evaluará en Fase 2 (referencia del proyecto)}

Para la Fase 2 (entrega: 7 de enero 2026), se implementará el núcleo en CUDA y se comparará contra CPU (y opcionalmente OpenMP)
con métricas:
\[
\text{Speedup } S=\frac{T_{\text{secuencial}}}{T_{\text{paralelo}}}
\qquad
\text{Efficiency (OpenMP) } E=\frac{S}{\#\text{núcleos}}.
\]
Además, se analizará en CUDA:
\begin{itemize}
  \item Configuración de \textit{blocks} e hilos por bloque.
  \item Coalescencia de memoria global.
  \item Uso de memoria compartida (tiling).
\end{itemize}


%\input{ejemplo}
%\input{subsections/07-anexos}
%\input{subsections/08-apendices}
%\input{subsections/09-glosario}

%% =====================================================
%% BIBLIOGRAFÍA
%% =====================================================
%\printreferences[Referencias Bibliográficas]

%% =====================================================
%% FIN DEL DOCUMENTO
%% =====================================================
\end{document}