\documentclass[10pt,a4paper]{article}

%% Carga del paquete de formato personalizado
\usepackage{formato}

%% =====================================================
%% CONFIGURACIÓN DE LOS DATOS DEL DOCUMENTO
%% =====================================================
\documenttype{Proyecto Final -- Fase 2}
\documenttitle{Paralelización de Algoritmos Matriciales Masivos con CUDA: Warshall Lógico (Cerradura Transitiva Booleana)}
\subject{Algoritmos Paralelos y Distribuidos}
\advisor{Mgt. Ray Dueñas Jiménez}
\semester{2025-II}

%% =====================================================
%% AUTORES
%% =====================================================
\addauthor{Castro Pari, Rayneld Fidel}
\addauthor{Mayhuire Chacon, Brenda Lucia}
\addauthor{Mendoza Quispe, Jose Daniel}
\addauthor{Perez Cahuana, Gabriel}
\addauthor{Zevallos Yanqui, Andy Jefferson}

%% =====================================================
%% MACROS ÚTILES (no requieren archivos externos)
%% =====================================================
% Inclusión segura de imágenes: si no existe el archivo, muestra un recuadro.
\newcommand{\maybeincludegraphics}[2][]{%
  \IfFileExists{#2}{\includegraphics[#1]{#2}}{%
    \fbox{\parbox[c][3.2cm][c]{0.92\linewidth}{\centering \textbf{Archivo no encontrado:}\\ \texttt{#2}}}%
  }%
}

% Inclusión segura de listings: si no existe el archivo, muestra un recuadro.
\newcommand{\maybeinputlisting}[2][]{%
  \IfFileExists{#2}{\lstinputlisting[#1]{#2}}{%
    \fbox{\parbox[c][3.2cm][c]{0.92\linewidth}{\centering \textbf{Archivo no encontrado:}\\ \texttt{#2}}}%
  }%
}

% Notación para indexar matriz plana
\newcommand{\idx}[3]{#1[#2\cdot N + #3]}

%% =====================================================
%% INICIO DEL DOCUMENTO
%% =====================================================
\begin{document}

%% =====================================================
%% PORTADA
%% =====================================================
\thispagestyle{empty}
\makeatletter
%% =====================================================
%% Portada - Documento académico - Versión APA
%% =====================================================
\begin{titlepage}
    \thispagestyle{empty}
    
    % Cambiar a interlineado sencillo solo para la portada
    \singlespacing
    
    %% =====================================================
    %% ENCABEZADO INSTITUCIONAL
    %% =====================================================
    \begin{center}
        UNIVERSIDAD NACIONAL DE SAN ANTONIO ABAD DEL CUSCO\\[0.3cm]
        FACULTAD DE INGENIERÍA ELÉCTRICA, ELECTRÓNICA, INFORMÁTICA Y MECÁNICA\\[0.3cm]
        ESCUELA PROFESIONAL DE INGENIERÍA INFORMÁTICA Y DE SISTEMAS
    \end{center}

    
    %% =====================================================
    %% ESCUDO/LOGO INSTITUCIONAL
    %% =====================================================
    \vspace{0.25cm}
    \begin{figure}[H]
        \centering
        \includegraphics[scale=0.3]{imagenes/escudo.png}
    \end{figure}
    
    %% =====================================================
    %% TÍTULO DEL DOCUMENTO
    %% =====================================================
    \begin{center}
        \Large{\textbf{\@documenttype}} \\[0.3cm]
        \rule{160mm}{0.6mm} \\[0.3cm]
        {\Large \textbf{\@documenttitle}} \\[0.1cm]
        \rule{160mm}{0.6mm}
    \end{center}
    
    \vspace{1cm}
    
    %% =====================================================
    %% INFORMACIÓN DEL CURSO Y PARTICIPANTES
    %% =====================================================
    \noindent\hspace*{2cm}%
    \parbox{0.5\textwidth}{
        \raggedright
        \textbf{Asignatura:} \\
        \hspace*{2em} \@subject \\[0.8cm]
        \textbf{Docente:} \\
        \hspace*{2em} \@advisor \\[0.8cm]
        % Condicional para singular/plural de "Estudiante(s)"
        \textbf{\ifnum\value{authorcount}>1 Estudiantes:\else Estudiante:\fi} \\
        \hspace*{2em}
        \parbox{\dimexpr\linewidth-2em\relax}{
            \raggedright
            \vspace{0.3em}
            \authorlist
        }
    }
    

    
    \vfill
    
    %% =====================================================
    %% FECHA DE PUBLICACIÓN
    %% =====================================================
    \begin{center}
        Cusco - Perú \\ \the\year
    \end{center}
    
    % Restaurar interlineado doble para el resto del documento
    \doublespacing
\end{titlepage}
\makeatother

%% =====================================================
%% PRELIMINARES
%% =====================================================
\startpreliminary
\clearpage

\tableofcontents
\clearpage

\listoffigures
\clearpage
\listoftables
\clearpage

%% =====================================================
%% PRINCIPAL
%% =====================================================
\startmain

%% =====================================================
\begin{abstractenv}
En la \textbf{Fase 2} del proyecto se implementan y evalúan versiones paralelas de la cerradura transitiva booleana
(\textbf{Warshall lógico}) utilizando dos enfoques: \textbf{OpenMP} en CPU (memoria compartida) y \textbf{CUDA} en GPU
(paralelismo masivo por hilos). El objetivo central es \textbf{comparar rendimiento} contra la versión secuencial de la Fase 1,
midiendo tiempos de ejecución para matrices masivas (recomendado $N \ge 1024$), y calculando métricas de \textbf{aceleración (speedup)}
y \textbf{eficiencia}.

La implementación OpenMP paraleliza el bucle de filas ($i$) en cada iteración $k$ usando \texttt{\#pragma omp parallel for},
mientras que la implementación CUDA mantiene $k$ secuencial y lanza, para cada $k$, un \textit{kernel} 2D que actualiza en paralelo
todas las celdas $(i,j)$ de la matriz. Se incluye validación opcional mediante una referencia BFS para tamaños pequeños,
y un diseño de experimentación reproducible (control de $N$, densidad $p$, semilla y repeticiones).

Finalmente, se incorporan plantillas de tablas para registrar \textbf{10 ejecuciones} por tamaño en OpenMP y CUDA, junto con una sección de
discusión que explica por qué CUDA suele superar a OpenMP (paralelismo, ancho de banda, ocultamiento de latencia y jerarquía de memoria).
\keywords{OpenMP, CUDA, GPU, CPU, rendimiento, speedup, eficiencia, matrices masivas, Warshall lógico}
\end{abstractenv}
\clearpage

%% =====================================================
\section{Introducción}
La evaluación de rendimiento es un paso obligatorio cuando se paralelizan algoritmos sobre matrices masivas, debido a que el tiempo total
depende tanto del \textbf{cómputo} como de los costos de \textbf{memoria} (caché en CPU, transferencias y jerarquía de memoria en GPU).
En la Fase 1 se implementó la versión secuencial y se dejó listo el escenario de pruebas; en esta \textbf{Fase 2} el foco pasa a:
\begin{itemize}
  \item construir una versión paralela en \textbf{CPU con OpenMP},
  \item construir una versión paralela en \textbf{GPU con CUDA},
  \item medir tiempos para distintos tamaños $N$ y reportar \textbf{métricas comparativas}.
\end{itemize}

La entrega incluye la demostración de ejecución, tablas de tiempos (10 repeticiones), cálculo de aceleración y una discusión técnica de
por qué CUDA suele ser más rápido que OpenMP para cargas matriciales intensivas.

%% =====================================================
\section{Objetivos}
\subsection{Objetivo general}
Implementar y evaluar el rendimiento de las versiones paralelas \textbf{OpenMP (CPU)} y \textbf{CUDA (GPU)} del núcleo de Warshall lógico,
comparándolas contra la versión secuencial, mediante tiempos de ejecución y métricas de aceleración/eficiencia para matrices masivas.

\subsection{Objetivos específicos}
\begin{itemize}
  \item Implementar la versión paralela en CPU usando \textbf{OpenMP}, definiendo el esquema de paralelización y su configuración de hilos.
  \item Implementar la versión paralela en GPU usando \textbf{CUDA}, definiendo el mapeo de hilos/bloques y parámetros de ejecución.
  \item Diseñar una metodología reproducible de medición: tamaños $N$, densidad $p$, semilla, 10 repeticiones y control de verificación.
  \item Registrar tiempos de ejecución y calcular \textbf{speedup} frente al secuencial y \textbf{eficiencia} para OpenMP.
  \item Analizar el rendimiento CUDA considerando bloques/hilos, acceso a memoria y coalescencia, y discutir diferencias con OpenMP.
\end{itemize}

%% =====================================================
\section{Marco teórico (solo OpenMP y CUDA)}

\subsection{OpenMP: paralelismo en memoria compartida}
OpenMP es un estándar para paralelización en CPU basado en directivas, pensado para arquitecturas de \textbf{memoria compartida}.
En este modelo, múltiples hilos acceden a los mismos arreglos en memoria principal y la performance depende de:
\begin{itemize}
  \item número de núcleos/hilos disponibles,
  \item balance de carga (\texttt{schedule(static/dynamic)}),
  \item eficiencia de caché (localidad espacial/temporal),
  \item ancho de banda de memoria y posibles conflictos (p.\ ej., \textit{false sharing}).
\end{itemize}

Una estructura común en OpenMP es paralelizar un bucle independiente:
\begin{center}
\texttt{\#pragma omp parallel for schedule(static)}
\end{center}
donde cada hilo procesa un subconjunto de iteraciones sin condiciones de carrera si no escriben en las mismas posiciones.

\subsection{CUDA: paralelismo masivo en GPU}
CUDA es un modelo de programación para GPU en el que el cómputo se expresa como \textit{kernels} ejecutados por miles de hilos.
Los hilos se organizan en:
\begin{itemize}
  \item \textbf{grid} (malla) de bloques,
  \item \textbf{bloques} de hilos,
  \item \textbf{hilos} que ejecutan el mismo kernel con distintos índices.
\end{itemize}

El rendimiento en CUDA está dominado por:
\begin{itemize}
  \item configuración de \textbf{threads por bloque} y \textbf{bloques totales},
  \item \textbf{ocupación} (cuántos warps activos por SM),
  \item \textbf{coalescencia} de accesos a memoria global,
  \item uso de jerarquía de memoria: \textbf{global}, \textbf{shared}, \textbf{constant} y cachés.
\end{itemize}

Para cómputo matricial, se busca que hilos contiguos accedan a posiciones contiguas para maximizar throughput de memoria.

%% =====================================================
\section{Implementación paralela (Fase 2)}

\subsection{Estrategia común y correctitud}
En ambas versiones paralelas, la iteración $k$ se mantiene \textbf{secuencial} (dependencia entre iteraciones). La ganancia se obtiene
paralelizando el trabajo dentro de cada $k$:
\begin{itemize}
  \item OpenMP: paralelismo por filas ($i$).
  \item CUDA: paralelismo 2D por celdas ($i,j$).
\end{itemize}

Para correctitud, se incluye un modo de verificación opcional usando una referencia BFS (recomendado solo para tamaños pequeños).

\subsection{Versión OpenMP (CPU)}
La versión OpenMP paraleliza el bucle de filas $i$ para cada $k$:
\begin{itemize}
  \item Cada hilo escribe únicamente la fila $i$ que le corresponde (sin carreras).
  \item Se aplica un atajo: si $A_{ik}=0$, no se procesa la fila $i$ en ese $k$.
  \item Se usa \texttt{schedule(static)} para repartir filas de forma uniforme.
\end{itemize}

\paragraph{Compilación y ejecución (Linux).}
\begin{itemize}
  \item Compilar: \texttt{gcc -O3 -std=c11 -fopenmp warshall\_menu\_omp.c -o warshall\_omp}
  \item Ejecutar: \texttt{./warshall\_omp 1024 0.05 1234 3 0 0}
\end{itemize}

\subsection{Versión CUDA (GPU)}
La versión CUDA realiza:
\begin{itemize}
  \item Copia inicial Host$\rightarrow$Device de la matriz.
  \item Para cada $k$, lanza un kernel 2D donde cada hilo actualiza una celda $(i,j)$.
  \item Copia final Device$\rightarrow$Host.
\end{itemize}

\paragraph{Configuración por defecto.}
\begin{itemize}
  \item Bloque: \texttt{dim3 block(16,16)}.
  \item Grid: \texttt{ceil(N/16) x ceil(N/16)}.
\end{itemize}

\paragraph{Compilación y ejecución (Linux).}
\begin{itemize}
  \item Compilar: \texttt{nvcc -O3 -std=c++17 warshall\_menu\_cuda.cu -o warshall\_cuda}
  \item Ejecutar: \texttt{./warshall\_cuda 1024 0.05 1234 3 0 0}
\end{itemize}

\subsection{Notas sobre medición de tiempo}
En OpenMP el tiempo medido corresponde al núcleo (llamada a \texttt{warshall\_logical\_omp}).
En CUDA, el tiempo medido (tal como está el código) incluye asignación, transferencias H2D/D2H y sincronizaciones por iteración.
Esto se reporta explícitamente para que la comparación sea transparente.

%% =====================================================
\section{Evaluación y métricas}

\subsection{Diseño experimental}
Se recomienda fijar:
\begin{itemize}
  \item tamaños $N$: 1024, 2048, 3072, 4096,
  \item densidad $p$: (ej.) 0.05 o la definida por el grupo,
  \item semilla: (ej.) 1234,
  \item repeticiones: 10 mediciones por cada $N$ (misma configuración).
\end{itemize}

\subsection{Tablas de tiempos (llenado manual)}
A continuación se dejan \textbf{dos tablas en blanco} para registrar \textbf{10 tiempos} por tamaño $N$:
una para OpenMP y una para CUDA. (Completar en segundos).

\begin{table}[H]
\centering
\caption{Tiempos de ejecución (10 repeticiones) -- OpenMP}
\resizebox{\textwidth}{!}{%
\begin{tabular}{c|cccccccccc|cc}
\hline
$N$ & $t_1$ & $t_2$ & $t_3$ & $t_4$ & $t_5$ & $t_6$ & $t_7$ & $t_8$ & $t_9$ & $t_{10}$ & $\bar{t}$ & $\min(t)$ \\
\hline
1024 &  &  &  &  &  &  &  &  &  &  &  &  \\
2048 &  &  &  &  &  &  &  &  &  &  &  &  \\
3072 &  &  &  &  &  &  &  &  &  &  &  &  \\
4096 &  &  &  &  &  &  &  &  &  &  &  &  \\
\hline
\end{tabular}}
\label{tab:times_omp}
\end{table}

\begin{table}[H]
\centering
\caption{Tiempos de ejecución (10 repeticiones) -- CUDA}
\resizebox{\textwidth}{!}{%
\begin{tabular}{c|cccccccccc|cc}
\hline
$N$ & $t_1$ & $t_2$ & $t_3$ & $t_4$ & $t_5$ & $t_6$ & $t_7$ & $t_8$ & $t_9$ & $t_{10}$ & $\bar{t}$ & $\min(t)$ \\
\hline
1024 &  &  &  &  &  &  &  &  &  &  &  &  \\
2048 &  &  &  &  &  &  &  &  &  &  &  &  \\
3072 &  &  &  &  &  &  &  &  &  &  &  &  \\
4096 &  &  &  &  &  &  &  &  &  &  &  &  \\
\hline
\end{tabular}}
\label{tab:times_cuda}
\end{table}

\subsection{Métricas requeridas}
Sea $T_{seq}$ el tiempo secuencial (de la Fase 1) y $T_{par}$ el tiempo paralelo (OpenMP o CUDA), para un mismo $N$.

\paragraph{Aceleración (Speedup).}
\[
S = \frac{T_{seq}}{T_{par}}
\]

\paragraph{Eficiencia (Efficiency) para OpenMP.}
\[
E = \frac{S}{\#\text{núcleos (o hilos)}}
\]

\subsection{Análisis del rendimiento (CUDA)}
Para CUDA se analiza:
\begin{itemize}
  \item configuración de bloques y threads por bloque (por defecto 16x16),
  \item accesos a memoria global y coalescencia (hilos contiguos en $j$),
  \item sincronización por iteración $k$ (en el código se fuerza \texttt{cudaDeviceSynchronize}).
\end{itemize}

%% =====================================================
\section{Gráficos (plantillas)}
Una vez completadas las tablas, se recomienda graficar:
\begin{itemize}
  \item Tiempo vs. $N$ (secuencial, OpenMP, CUDA).
  \item Speedup vs. $N$ (OpenMP y CUDA respecto al secuencial).
\end{itemize}



%% =====================================================
\section{Discusión: ¿por qué CUDA suele ser más rápido que OpenMP?}
En general, CUDA tiende a superar a OpenMP en este tipo de cargas matriciales por varias razones:

\subsection{Paralelismo disponible}
OpenMP está limitado por el número de núcleos de CPU (decenas de hilos como máximo en un equipo típico),
mientras que CUDA explota \textbf{miles de hilos} concurrentes. En Warshall lógico, cada iteración $k$ puede actualizar
$N^2$ celdas, lo que se ajusta de forma natural al paralelismo masivo de GPU.

\subsection{Ancho de banda y ocultamiento de latencia}
Las GPU están diseñadas para alto throughput: cuando un grupo de hilos (warp) espera memoria, el SM puede ejecutar otros warps,
ocultando latencias. En CPU, aunque existen cachés y vectorización, el rendimiento suele quedar limitado por el ancho de banda
y la presión de memoria al recorrer matrices grandes.

\subsection{Accesos a memoria y coalescencia}
En CUDA, si los hilos contiguos acceden a posiciones contiguas (por ejemplo, celdas con $j$ consecutivo), las lecturas/escrituras
pueden coalescerse, aprovechando transacciones eficientes en memoria global. En OpenMP, cada hilo recorre secuencialmente $j$ en su fila;
aunque eso favorece localidad, el paralelismo global es menor y la ganancia se estanca al saturar caché/memoria.

\subsection{Overhead y casos donde OpenMP compite}
CUDA tiene overhead de lanzamiento de kernels y transferencias Host--Device (si se miden). Para tamaños moderados o si el cálculo
no domina el costo total, OpenMP puede ser competitivo. Para $N$ grande (carga cúbica), el cómputo domina y CUDA suele despegar.

\subsection{Oportunidades de mejora en CUDA (memoria compartida)}
La versión actual usa principalmente memoria global. Una optimización típica es \textit{tiling} con \textbf{shared memory} para reutilizar
segmentos de la fila $k$ (y mejorar el acceso repetido), además de evaluar diferentes configuraciones de bloque para mejorar ocupación.

%% =====================================================
\section{Conclusiones}
\begin{itemize}
  \item Se implementaron dos versiones paralelas del núcleo: OpenMP (CPU) y CUDA (GPU), manteniendo $k$ secuencial.
  \item Se dejó un protocolo de medición reproducible con tablas para 10 ejecuciones por tamaño, base para speedup y eficiencia.
  \item La discusión técnica muestra por qué CUDA suele lograr mayor aceleración en cargas matriciales masivas: más paralelismo y throughput.
\end{itemize}

\section{Trabajo futuro}
\begin{itemize}
  \item Medir el tiempo CUDA separando cómputo (kernel) de transferencias usando \texttt{cudaEvent}.
  \item Probar variantes de bloque (p.\ ej., 8x8, 16x16, 32x8) y reportar su impacto.
  \item Implementar \textit{tiling} con shared memory para reutilización de datos en cada $k$.
\end{itemize}

%% =====================================================
\begin{thebibliography}{9}
\bibitem{openmp}
OpenMP Architecture Review Board,
\textit{OpenMP Application Programming Interface (Specification)}.

\bibitem{cuda_guide}
NVIDIA,
\textit{CUDA C++ Programming Guide}.

\bibitem{cuda_best_practices}
NVIDIA,
\textit{CUDA C++ Best Practices Guide}.
\end{thebibliography}

%% =====================================================
\appendix
\section{Código OpenMP (warshall\_menu\_omp.c)}
\maybeinputlisting[
  language=C,
  caption={\texttt{warshall\_menu\_omp.c} -- Warshall lógico con OpenMP + menú + medición},
  label={lst:warshall_omp}
]{warshall_menu_omp.c}

\section{Código CUDA (warshall\_menu\_cuda.cu)}
\maybeinputlisting[
  language=C++,
  caption={\texttt{warshall\_menu\_cuda.cu} -- Warshall lógico con CUDA + menú + medición},
  label={lst:warshall_cuda}
]{warshall_menu_cuda.cu}

\end{document}